%-----------------------------------------------------------------------
%      Start introduction here and continue with additional sections
%-----------------------------------------------------------------------

\section{Introduction}

This is the place for introduction.

\subsection{Subsection}

Example text:

We shall consider the specific transport property $\Q$ and note that its spatial and temporal variation is governed by a second-order partical differential equation (PDE), viz.\
\begin{align}
    \ddt{}(\rho \Q) + \div(\rho \Q \U) - \Gamma_{\Q}\laplacian\Q - S_\Q(\Q) = 0.
    \label{eq:genTransEq}
\end{align}
Herein, $\Q=\Q(\xt)$ is an arbitrary general intensive physical quantitity, e.g., a fluid property (scalar or tensor of any rank). Thus, \autoref{eq:genTransEq} is often referred to as generic transport equation.

\OF (Open Field Operation And Manipulation) is a flexible and mature C++ Class Library for Computational Continuum Mechanics (CCM) and Multiphysics. Its Object-Oriented-Programming (OOP) paradigm enables to \emph{mimic data types and basic operations} of CCM using top-level syntax as close as possible to the conventional mathematical notation \emph{for tensors and partial differential equations}:
\begin{lstlisting}[emph={ddt,div,laplacian}]
solve
(
  fvm::ddt(rho,Phi)
  + fvm::div(phi, Phi)
  - fvm::laplacian(Gamma, Phi)
 ==
  Sphi
);
\end{lstlisting}
Beside providing \OF code itself, spatial and temporal discretisation of \autoref{eq:genTransEq} can be also described in a precise and concise manner using the finite-volume notation\, \cite{Weller1998} - see \autoref{tab:FiniteVolumeNotation}.

\begin{table}
        \caption{Finite Volume Notation}
        \label{tab:FiniteVolumeNotation}
        \centering
                \begin{tabular}{p{0.3\textwidth}p{0.3\textwidth}}
                  \toprule
                          \multicolumn{2}{l}{implicit differential operators}\\
                  \midrule
                          rate of change     & $\fvmddt{\rho\q}$ \\%time derivative
                          convection term    & $\fvmdiv{F}{\q}{S}{\genFactor}$ \\
                          diffusion term     & $\fvmlaplacian{\Gamma}{\q}$ \\
                          linear part of source term & $\fvmSp{S_p}{\q}$ \\
                        \hline
                          \multicolumn{2}{l}{explicit differential operators}\\
                        \hline
                          temporal term      & $\fvcddt{\rho\q}$ \\
                          divergence term    & $\fvcdiv{\rho\U}{\q}{S}{\genFactor}$ \\
                          laplacian term     & $\fvclaplacian{\Gamma}{\q}$ \\
                          constant part of source term & $S_u$\\
                  \bottomrule
                \end{tabular}
\end{table}
